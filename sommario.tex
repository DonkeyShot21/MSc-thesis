\newpage
\chapter*{Sommario}

\addcontentsline{toc}{chapter}{Sommario}

% \noindent Questo elaborato combina elementi di fisica solare osservativa ed informatica all'avanguardia. La forte influenza del Sole sull'ambiente terrestre rende necessario monitorare e prevedere la sua attivit\`{a}. Le macchie solari, manifestazioni di forti perturbazioni nel campo magnetico del Sole, sono una delle caratteristiche visibili che possono essere studiate per modellizzare i cicli solari. Finora, il conteggio delle macchie solari \`{e} stato per lo pi\`{u} fatto da esseri umani e la comunit\`{a} scientifica sembra riluttante all'introduzione di algoritmi che creerebbero discontinuit\`{a} con i metodi di osservazione tradizionali. Lo scopo di questa tesi \`{e} dimostrare che, utilizzando il deep learning, \`{e} possibile costruire un programma in grado di apprendere da scienziati esperti e di eseguire automaticamente l'annotazione di immagini solari secondo criteri umani. Alcuni test sono stati progettati per valutare la qualit\`{a} della soluzione proposta rispetto alle prestazioni umane medie. I risultati sono promettenti e mostrano che l'algoritmo riesce a catturare l'andamento del ciclo solare, rendendolo un valido strumento per la stima dell'attivit\`{a} del Sole.


\noindent La forte influenza del Sole sull'ambiente terrestre rende infatti necessario monitorare e prevedere la sua attivit\`{a}. Le macchie solari, manifestazioni di forti perturbazioni nel campo magnetico del Sole, sono una delle caratteristiche visibili che possono essere studiate per modellizzare i cicli solari. Finora, il conteggio delle macchie solari \`{e} stato per lo pi\`{u} eseguito da esseri umani e la comunit\`{a} scientifica sembra riluttante all'utilizzo di algoritmi che, se introdotti, creerebbero discontinuit\`{a} con i metodi di osservazione tradizionali. Lo scopo di questa tesi, che combina elementi di fisica solare osservativa ed informatica all'avanguardia, \`{e} dimostrare che, utilizzando il deep learning, \`{e} possibile costruire un programma che, se opportunamente addestrato, \`{e} in grado di apprendere da scienziati esperti ed eseguire automaticamente l'annotazione di immagini solari secondo criteri umani. Alcuni test sono stati progettati per valutare la qualit\`{a} delle soluzioni proposte dal programma, rispetto alle prestazioni umane medie. I risultati sono promettenti e mostrano che l'algoritmo riesce a cogliere l'andamento del ciclo solare, rendendolo un valido strumento per la stima dell'attivit\`{a} del Sole.
