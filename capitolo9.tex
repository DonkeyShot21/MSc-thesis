\chapter{Conclusion}
\label{capitolo9}
\thispagestyle{empty}
\noindent This work demonstrated that, using deep learning, it is possible to detect and group sunspots the way humans do it. This is important because it solves the problem of discontinuities in the measurements that other state of the art methods introduce. Our algorithm tries to replicate human understanding of sunspots. Doing so, it greatly reduces the bias in the annotation criteria, with respect to normal image processing methods. Also, the model we propose can be trained to replicate any annotation style produced by humans, in the sense that it is potentially able to learn the subjectivity of the annotator.\\
The other interesting contribution is that our algorithm performs image annotation and sunspot counting from the image without human supervision, additional data or complementary programs. As pointed out in chapter~\autoref{capitolo3}, most state of the art algorithms are more specialized, in the sense that they are able either to segment sunspots or they use the magnetogram to find the groups. Our solution does both things, without even using magnetic data.\\
Apart from the novelties of the work presented in this thesis, what makes it even more valuable are the opportunities of learning it provided. The day I arrived at the European Space Agency, I had little to no experience with solar observation. Now I am able to operate a telescope and fix it when it periodically stops working. I learned about most of the aspects of solar physics and I understood why modeling our dear star is so important. Similarly, this thesis was also a journey inside deep learning. When I started, I already had some knowledge of machine learning, but I still had to discover the real power of neural networks.
