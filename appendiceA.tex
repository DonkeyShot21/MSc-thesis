\chapter{Glossary}
\label{appendiceA}
\thispagestyle{empty}

The definitions in this glossary are taken from NOAA's space weather glossary \cite{NOAAglossary}.\\


\newenvironment{myindentpar}[1]%
{\begin{list}{}%
        {\setlength{\leftmargin}{#1}}%
        \item[]%
}
{\end{list}}


\textbf{Active Region:}
\begin{myindentpar}{1cm}
  A localized, transient volume of the solar atmosphere in which plages, sunspots, faculae, flares, etc. may be observed.
\end{myindentpar}

\textbf{Aurora:}
\begin{myindentpar}{1cm}
A faint visual (optical) phenomenon on the Earth associated with geomagnetic activity, which occurs mainly in the high-latitude night sky. Typical auroras are 100 to 250 km above the ground. The Aurora Borealis occurs in the northern hemisphere and the Aurora Australis occurs in the southern hemisphere.
\end{myindentpar}

\textbf{Corona:}
\begin{myindentpar}{1cm}
  The outermost layer of the solar atmosphere, characterized by low densities and extraordinarily high temperatures that extends to several solar radii. The heating of the corona is still a mystery. The shape of the corona is different at solar maximum and at solar minimum.
\end{myindentpar}

\textbf{Coronal Hole:}
\begin{myindentpar}{1cm}
  An extended region of the corona, exceptionally low in density (large open "gaps"), and associated with photospheric regions. Coronal holes are closely associated with those regions on the Sun that have an "open" magnetic geometry, that is, the magnetic field lines associated with them extend far outward into interplanetary space, rather than looping back to the photosphere. Ionized material can flow easily along such a path, and this in turn aids the mechanism that causes high speed solar wind streams to develop.
\end{myindentpar}

\textbf{Coronal Mass Ejection:}
\begin{myindentpar}{1cm}
  An observable change in coronal structure that occurs on a time scale between a few minutes and several hours, and involves the appearance of a new discrete, bright, white light feature in the coronagraph field of view, that displays a predominantly outward motion. The solar corona material is massive in size (they can occupy up to a quarter of the solar limb), and frequently accompanied by the remnants of an eruptive prominence, and less often by a strong solar flare. The leading edges of fast-moving CMEs drive giant shock waves before them through the solar wind at speeds up to 1200 km per second. Some astronomers believe that CMEs are the crucial link between a solar disturbance its propagation through the heliosphere, and the effects on the Earth.
\end{myindentpar}

\textbf{Differential Rotation:}
\begin{myindentpar}{1cm}
  The change in solar rotation rate with latitude. Low latitudes rotate at a faster angular rate (approx. 14 degrees per day) than do high latitudes (approx. 12 degrees per day). For example, the equatorial rotation period is 27.7 days compared to 28.6 days at latitude 40 degrees.
\end{myindentpar}

\textbf{Disk:}
\begin{myindentpar}{1cm}
  The visible surface of the Sun (or any heavenly body) projected against the sky.
\end{myindentpar}

\textbf{Eruptive:}
\begin{myindentpar}{1cm}
  Solar activity levels with at least one radio event and several chromospheric events per day.
\end{myindentpar}

\textbf{Facula:}
\begin{myindentpar}{1cm}
  A bright cloud-like feature located a few hundred km above the photosphere near sunspot groups, seen in white light. Facula are seldom visible except near the solar limb, although they occur all across the Sun. Facula are clouds of emission that occur where a strong magnetic field creates extra heat (about 300 degrees K above surrounding areas).
\end{myindentpar}

\textbf{Filament:}
\begin{myindentpar}{1cm}
  A mass of gas suspended over the photosphere by magnetic fields and seen as dark lines threaded over the solar disk. A filament on the limb of the Sun seen in emission against the dark sky is called a prominence.
\end{myindentpar}

\textbf{Flux:}
\begin{myindentpar}{1cm}
  The rate of flow of a physical quantity through a reference surface.
\end{myindentpar}

\textbf{Geomagnetic Field:}
\begin{myindentpar}{1cm}
  The magnetic field observed in and around the Earth. The intensity of the magnetic field at the Earth's surface is approximately 0.32 gauss at the equator and 0.62 gauss at the north pole.
\end{myindentpar}

\textbf{H-alpha:}
\begin{myindentpar}{1cm}
  This absorption line of neutral hydrogen falls in the red part of the visible spectrum and is convenient for solar observations. The H-alpha line is universally used for observations of solar flares and prominences.
\end{myindentpar}

\textbf{Helioseismology:}
\begin{myindentpar}{1cm}
  A method for studing the Sun by utilizing waves that propagate throughout the star to measure its invisible internal structure and dynamics.
\end{myindentpar}

\textbf{Limb:}
\begin{myindentpar}{1cm}
  The edge of the solar disk.
\end{myindentpar}

\textbf{Magnetogram:}
\begin{myindentpar}{1cm}
  Solar magnetograms are a graphic representation of solar magnetic field strengths and polarity.
\end{myindentpar}

\textbf{Penumbra:}
\begin{myindentpar}{1cm}
  The sunspot area that may surround the darker umbra or umbrae. It consists of linear bright and dark elements radial from the sunspot umbra.
\end{myindentpar}

\textbf{Plasma:}
\begin{myindentpar}{1cm}
  Any ionized gas, that is, any gas containing ions and electrons.
\end{myindentpar}

\textbf{Prominence:}
\begin{myindentpar}{1cm}
  A term identifying cloud-like features in the solar atmosphere. The features appear as bright structures in the corona above the solar limb and as dark filaments when seen projected against the solar disk.
\end{myindentpar}

\textbf{Quiet:}
\begin{myindentpar}{1cm}
  Solar activity levels with less than one chromospheric event per day.
\end{myindentpar}

\textbf{Solar Cycle:}
\begin{myindentpar}{1cm}
  The approximately 11-year quasi-periodic variation in frequency or number of solar active events.
\end{myindentpar}

\textbf{Solar Maximum:}
\begin{myindentpar}{1cm}
  The month(s) during the solar cycle when the 12-month mean of monthly average sunspot numbers reaches a maximum.
\end{myindentpar}

\textbf{Solar Minimum:}
\begin{myindentpar}{1cm}
  The month(s) during the solar cycle when the 12-month mean of monthly average sunspot numbers reaches a minimum.
\end{myindentpar}

\textbf{Umbra:}
\begin{myindentpar}{1cm}
  The dark core or cores (umbrae) in a sunspot with penumbra, or a sunspot lacking penumbra.
\end{myindentpar}

\textbf{White-Light:}
\begin{myindentpar}{1cm}
  Sunlight integrated over the visible portion of the spectrum (4000 - 7000 angstroms) so that all colors are blended to appear white to the eye.
\end{myindentpar}
