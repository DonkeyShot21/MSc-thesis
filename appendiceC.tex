\chapter{Helios Processing Pipeline}
\label{appendiceC}
\thispagestyle{empty}
This appendix describes the first part of the work that was done at the Helios observatory of the European Space Agency. The observatory belongs to the CESAR (Cooperation through Education in Science and Astronomy Research) initiative, whose objective is to provide students from european secondary schools and universities with hands-on experience in Astronomy research in general and in Radio Astronomy and Optical Astronomy in particular. Helios was installed at ESAC (European Space Astronomy Centre) in 2012 and includes two telescopes (Figure wanama):
\begin{itemize}
  \item Coronado Solarmax II 90, H-alpha, double stack, with specifications:
  \begin{itemize}
    \item Aperture: 90mm
    \item Focal Length: 800mm
    \item Bandwidth: \textless0.5 \AA
  \end{itemize}
  \item Bresser AR-102, visible (white-light), with specifications:
  \begin{itemize}
    \item Aperture: 102mm
    \item Focal Length: 1000mm
    \item Solar Filter: BAADER AstroSolar Safety Filter
  \end{itemize}
\end{itemize}
\bigbreak
The two telescopes are mounted on the same robotic arm (Figure wanama) and are very different from each other. The main difference is in the type of filter they have on board. On the one hand, the filter that comes with the Coronado Solarmax isolates the H-alpha band, a deep-red visible spectral line. H-alpha is particularly useful in solar astronomy because it enables the observation of the atmosphere of the Sun. On the other hand, the filters that are used for white-light observation do not reduce the portion of the specturm that enters the scope, but rather decrease the intesity of the radiation. Therefore each pixel of the sensor, attached at the bottom of the scope, receives the whole visible spectrum and integrates it to obtain the total intensity.
\bigbreak
\begin{figure}[t!]
    \centering
    \captionsetup{justification=centering}
    \includegraphics[height=0.4\textheight]{./pictures/solar-telescopes}
    \caption{A picture of the two telescopes}
    \label{fig:pred-vs-ground}
\end{figure}
Given these differences, dedicated processing techniques are used for each telescope. Nonetheless, the preliminary steps of the two pipelines are shared, although with different parameters, and they are therefore presented together. In fact, we can logically divide the pipelines in two phases (which are treated in the next sections):
\begin{itemize}
  \item \textbf{Preliminary adjustments}, performed for both telescopes with different parameters;
  \item \textbf{Feature enhancement}, dedicated for each telescope.
\end{itemize}

\section{Preliminary Adjustments}
The first phase of the processing starts with a set of quality control checks. For each image we need to make sure that they are not corrupted or completely black, since files could be damaged during the transfer from the observatory to the servers that are allocated for the processing. Then, a second check, called cloud control, is performed to retain only the images where the Sun is perfectly visible, and discard the cloudy ones.
\bigbreak
Once it is certain that the images are up to standard, the pipeline proceeds to perform some corrections that are necessary to remove the artifacts introduced by the sensor. In fact, even the best sensors suffer from variations in the pixel sensitivity of the detector and distortions in the optical path. Luckily, is possible to account for these errors unsing two techniques that are really popular in digital photograpy: \textbf{dark-frame subtraction} and \textbf{flat-field correction}. The former tries to minimize the noise due to defective pixels and dark currents, while the latter adjusts for the relative sensitivity of pixels. The dark-frame is created by taking long exposure images in con complete darkness, or, in this case, when the lid of the telescope is on. The flat-field, instead, is generated attaching an omogeneous light source over the instrument, to measure how much each pixel reacts to it. Thus, two calibration images are used to adjust the images with the following formula:
\begin{equation}
C = \frac{(R - D)}{(F - D)}
\end{equation}
where $C$ is the corrected image, $R$ is the raw image, and $F$ and $D$ are the flat-field and the dark-frame respectively.
\bigbreak
The third subphase of the preliminary adjustments phase deals with image centering and axial tilt correction. Although, as explained in Appendix D, the telescope itself and the master control program already account for the tracking of the Sun, the limited accuracy of mechanical arms makes it necessary to use image processing techniques to align and center the solar disk to the frame. Moreover, in this case, centering the image is made easier by the fact that the shape of the object we are looking for is known and its size can be calculated with simple geometry. In fact, first, the sun can be considered a sphere, since its oblateness is neglectable; and second, the size of the appearent radius can be determined from the Sun-Earth distance. For the latter calculation it is sufficent to know the radius at one point of the orbit (we used the perihelion), and remember that the appearent size of an object is proportional to its distance. Therefore the radius can be calculated as:
\begin{equation}
  R = R_p\frac{d}{d_p}
\end{equation}
where $R$ is the desired radius in pixels at distance $d$, and $R_p$ and $d_p$ are respectively the radius and the distance at perihelion.
\bigbreak
Knowing the appearent radius for each day, it is possible to build a dynamic template of the disk, and template matching can be applied to find the center of the Sun in pixel coordinates. Also, the radius can be used to determine the size of the patch to be cropped in order to obtain an image in which the Sun is centered. The perfect alignment is achieved by rotating the resulting image by an angle that is equal to the axial tilt.

\section{Feature Enhancement}
\subsection{White-light}















- mean and std dev
- gradient sharpening
- limb darkening correction




\subsection{H-alpha}

- off limb emission enhancement
- kernel sharpening

% Given these differences, dedicated processing techniques are used to extract specific features from the raw images of each telescope.  In fact, the two pipelines
























% end
